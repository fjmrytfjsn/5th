\documentclass[dvipdfmx, 11pt]{jsarticle}

% \usepackage[utf8]{inputenc}
\usepackage[dvipdfmx]{graphicx}
\usepackage[top=25truemm, bottom=25truemm, left=20truemm, right=20truemm]{geometry}
\usepackage{okumacro}
\usepackage{diagbox}
\usepackage{amsmath, amssymb}
\usepackage{enumerate}
\usepackage{siunitx}
\usepackage{url}
\usepackage{pdfpages}
\usepackage{here}
\usepackage{slashbox}
\usepackage{diagbox}


\setlength\textfloatsep{2truemm}

%---------------------------------------------------------------------

% \fontsize{11ptj}{16ptj}\selectfont
\setlength{\baselineskip}{16pt}
\setlength{\columnsep}{5mm}

%---------------------------------------------------------------------

%表
\usepackage{tabularx}
\newcolumntype{Y}{&gt;{\centering\arraybackslash}X}

%---------------------------------------------------------------------

\usepackage{listings,jvlisting}
\lstset{
    basicstyle={\ttfamily},
    identifierstyle={\small},
    commentstyle={\smallitshape},
    keywordstyle={\small\bfseries},
    ndkeywordstyle={\small},
    stringstyle={\small\ttfamily},
    frame={tb},
    tabsize=4,
    breaklines=true,
    columns=[l]{fullflexible},
    numbers=left,
    xrightmargin=0zw,
    xleftmargin=3zw,
    numberstyle={\scriptsize},
    stepnumber=1,
    numbersep=1zw,
    lineskip=-0.5ex
}
\newcommand{\prog}[3]{
    \lstinputlisting[label=code:#1, caption=#2]{materials/#3}
}
\renewcommand{\lstlistingname}{コード}

%---------------------------------------------------------------------
\makeatletter
\def\Hline{
    \noalign{\ifnum0=`}\fi\hrule \@height 2pt \futurelet
    \reserved@a\@xhline
}
\makeatother

%表の空欄
\newcommand{\blank}{\textbf{---}}

%数式(番号付き)
\newcommand{\eq}[1]{
    \begin{eqnarray}
        #1
    \end{eqnarray}
}
%数式(番号無し)
\newcommand{\EQ}[1]{
    \begin{eqnarray*}
        #1
    \end{eqnarray*}
}

%一階微分
\newcommand{\diff}[2]{\frac{d #1}{d #2}}
%二階微分
\newcommand{\DIFF}[2]{\frac{d^2 #1}{d {#2}^2}}

%図(1つ)
\newcommand{\fig}[4][0.9]{
    \begin{figure}[H]
        \centering
        \includegraphics[width=#1\linewidth]{materials/#2}
        \caption{#3}
        \label{fig:#4}
    \end{figure}
}
%図(2つ)
\newcommand{\subfig}[6]{
    \begin{figure}[H]
        \centering
        \begin{minipage}{0.45\linewidth}
            \centering
            \includegraphics[width=\linewidth]{materials/#1}
            \caption{#2}
            \label{fig:#3}
        \end{minipage}
        \begin{minipage}{0.45\linewidth}
            \centering
            \includegraphics[width=\linewidth]{materials/#4}
            \caption{#5}
            \label{fig:#6}
        \end{minipage}
    \end{figure}
}
%図(3つ)
\newcommand{\thirdfig}[9]{
    \begin{figure}[H]
        \begin{minipage}{0.3\linewidth}
            \centering
            \includegraphics[width=\linewidth]{materials/#1}
            \caption{#2}
            \label{fig:#3}
        \end{minipage}
        \begin{minipage}{0.3\linewidth}
            \centering
            \includegraphics[width=\linewidth]{materials/#4}
            \caption{#5}
            \label{fig:#6}
        \end{minipage}
        \begin{minipage}{0.3\linewidth}
            \centering
            \includegraphics[width=\linewidth]{materials/#7}
            \caption{#8}
            \label{fig:#9}
        \end{minipage}
    \end{figure}
}

% \pagestyle{empty}

\begin{document}

\begin{titlepage}
    \centering
    \vspace*{\stretch{3}}
    \textbf{令和4年度 第5学年 情報工学実験II}
    \vspace{40pt} \\
    {\LARGE 明石高専学生食堂システム}
    \vspace{10pt} \\
    {\LARGE 要求定義書}
    \vspace{80pt} \\
    チーム9
    \vspace{10pt} \\
    \begin{tabular}{lll}
        作成者      & E1832 & 藤村勇仁 \\
                    &       & \\
        共同作成者  & E18 &  \\
                    & E18 &  \\
                    &       & \\
        作成日      & \multicolumn{2}{l}{2022年4月26日(火)}
    \end{tabular}
    \vspace{\baselineskip} \\
    \vspace{\stretch{4}}
\end{titlepage}

\section{システム化の目的}
    現在の明石高専のシステムでは,食堂のメニューはHPから新着情報のページへ行き,そこから食堂のメニューの更新情報を探さなければ見ることができない.また,売り切れ情報は直接食堂へ行き券売機を見なければいけない. \par
    そこで,スマートフォンのブラウザ上で Aセット,Bセット,常設メニューとそれらの価格を表示し,学生が売り切れ情報を入力することで売り切れ情報を共有するソフトウェアを開発する.

\section{システムの機能要件}
    \begin{itemize}
        \item スマートフォンのブラウザ上で動作すること.
        \item 対象とする利用者は「ユーザ」と「管理者」とする. \\
            ただし,「管理者」は利用者に含めなくてもよい.
        \item 対象とするメニューは明石高専学生食堂のAセット,Bセット,常設メニューとする.
        \item AセットとBセットは各日のメニューを表示できること.
        \item 各メニューの価格が表示されること.
        \item Aセット,Bセット,常設メニューで売り切れたものがあればその情報を入力できること. \\
            ここで,売り切れ情報を入力するのは「ユーザ」とする.
        \item 売り切れ情報を速やかに反映できること. \\
            これはページの再読込による情報更新でも,プッシュ型の情報更新でもよい.
        \item WebサーバはApacheとする. \\
            ただし,プッシュ型の情報更新をする場合,Apacheを使わなくてもよい.
        \item サーバサイド技術はPHP,PythonによるCGI,Python+フレームワークFlaskのいずれ
        かとする.
        \item データベースはPostgreSQLとする.
    \end{itemize}

\section{データフローダイアグラム}
    図\ref{fig:dfd}にデータフローダイアグラムを示す.
    \fig[]{dfd.pdf}{データフローダイアグラム}{dfd}

\section{データディクショナリ}


\section{開発スケジュール}
    開発スケジュールは以下のようになっている.
    \begin{itemize}
        \item 要求分析と要求定義 \\       
            $\rightarrow$ 第2週〜第4週
        \item 外部設計と内部設計 \\
            $\rightarrow$ 第5週〜第7週
        \item プログラミング \\
            $\rightarrow$ 第8週〜第13週
        \item テストとドキュメント作成 \\
            $\rightarrow$ 第14週
    \end{itemize}

\section{各段階におけるリーダ}
    各段階におけるリーダは以下のようになっている.
    \begin{itemize}
        \item 要求分析と要求定義 \\       
            $\rightarrow$ 首浦
        \item 外部設計と内部設計 \\
            $\rightarrow$ 藤村
        \item プログラミング \\
            $\rightarrow$ 土井
        \item テストとドキュメント作成 \\
            $\rightarrow$ 首浦
    \end{itemize}

% \bibliographystyle{jplain}
% \bibliography{reference.bib}

\end{document}
