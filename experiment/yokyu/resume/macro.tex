% \usepackage[utf8]{inputenc}
\usepackage[dvipdfmx]{graphicx}
\usepackage[top=25truemm, bottom=25truemm, left=20truemm, right=20truemm]{geometry}
\usepackage{okumacro}
\usepackage{diagbox}
\usepackage{amsmath, amssymb}
\usepackage{enumerate}
\usepackage{siunitx}
\usepackage{url}
\usepackage{pdfpages}
\usepackage{here}
\usepackage{slashbox}
\usepackage{diagbox}


\setlength\textfloatsep{2truemm}

%---------------------------------------------------------------------

% \fontsize{11ptj}{16ptj}\selectfont
\setlength{\baselineskip}{16pt}
\setlength{\columnsep}{5mm}

%---------------------------------------------------------------------

%表
\usepackage{tabularx}
\newcolumntype{Y}{&gt;{\centering\arraybackslash}X}

%---------------------------------------------------------------------

\usepackage{listings,jvlisting}
\lstset{
    basicstyle={\ttfamily},
    identifierstyle={\small},
    commentstyle={\smallitshape},
    keywordstyle={\small\bfseries},
    ndkeywordstyle={\small},
    stringstyle={\small\ttfamily},
    frame={tb},
    tabsize=4,
    breaklines=true,
    columns=[l]{fullflexible},
    numbers=left,
    xrightmargin=0zw,
    xleftmargin=3zw,
    numberstyle={\scriptsize},
    stepnumber=1,
    numbersep=1zw,
    lineskip=-0.5ex
}
\newcommand{\prog}[3]{
    \lstinputlisting[label=code:#1, caption=#2]{materials/#3}
}
\renewcommand{\lstlistingname}{コード}

%---------------------------------------------------------------------
\makeatletter
\def\Hline{
    \noalign{\ifnum0=`}\fi\hrule \@height 2pt \futurelet
    \reserved@a\@xhline
}
\makeatother

%表の空欄
\newcommand{\blank}{\textbf{---}}

%数式(番号付き)
\newcommand{\eq}[1]{
    \begin{eqnarray}
        #1
    \end{eqnarray}
}
%数式(番号無し)
\newcommand{\EQ}[1]{
    \begin{eqnarray*}
        #1
    \end{eqnarray*}
}

%一階微分
\newcommand{\diff}[2]{\frac{d #1}{d #2}}
%二階微分
\newcommand{\DIFF}[2]{\frac{d^2 #1}{d {#2}^2}}

%図(1つ)
\newcommand{\fig}[4][0.9]{
    \begin{figure}[H]
        \centering
        \includegraphics[width=#1\linewidth]{materials/#2}
        \caption{#3}
        \label{fig:#4}
    \end{figure}
}
%図(2つ)
\newcommand{\subfig}[6]{
    \begin{figure}[H]
        \centering
        \begin{minipage}{0.45\linewidth}
            \centering
            \includegraphics[width=\linewidth]{materials/#1}
            \caption{#2}
            \label{fig:#3}
        \end{minipage}
        \begin{minipage}{0.45\linewidth}
            \centering
            \includegraphics[width=\linewidth]{materials/#4}
            \caption{#5}
            \label{fig:#6}
        \end{minipage}
    \end{figure}
}
%図(3つ)
\newcommand{\thirdfig}[9]{
    \begin{figure}[H]
        \begin{minipage}{0.3\linewidth}
            \centering
            \includegraphics[width=\linewidth]{materials/#1}
            \caption{#2}
            \label{fig:#3}
        \end{minipage}
        \begin{minipage}{0.3\linewidth}
            \centering
            \includegraphics[width=\linewidth]{materials/#4}
            \caption{#5}
            \label{fig:#6}
        \end{minipage}
        \begin{minipage}{0.3\linewidth}
            \centering
            \includegraphics[width=\linewidth]{materials/#7}
            \caption{#8}
            \label{fig:#9}
        \end{minipage}
    \end{figure}
}
