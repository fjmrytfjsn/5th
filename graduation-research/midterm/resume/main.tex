\documentclass[dvipdfmx, twocolumn, 11pt]{jsarticle}

% \usepackage[utf8]{inputenc}
\usepackage[dvipdfmx]{graphicx}
\usepackage[top=25truemm, bottom=25truemm, left=20truemm, right=20truemm]{geometry}
\usepackage{okumacro}
\usepackage{diagbox}
\usepackage{amsmath, amssymb}
\usepackage{enumerate}
\usepackage{siunitx}
\usepackage{url}
\usepackage{pdfpages}
\usepackage{here}
\usepackage{slashbox}
\usepackage{diagbox}


\setlength\textfloatsep{2truemm}

%---------------------------------------------------------------------

% \fontsize{11ptj}{16ptj}\selectfont
\setlength{\baselineskip}{16pt}
\setlength{\columnsep}{5mm}

%---------------------------------------------------------------------

%表
\usepackage{tabularx}
\newcolumntype{Y}{&gt;{\centering\arraybackslash}X}

%---------------------------------------------------------------------

\usepackage{listings,jvlisting}
\lstset{
    basicstyle={\ttfamily},
    identifierstyle={\small},
    commentstyle={\smallitshape},
    keywordstyle={\small\bfseries},
    ndkeywordstyle={\small},
    stringstyle={\small\ttfamily},
    frame={tb},
    tabsize=4,
    breaklines=true,
    columns=[l]{fullflexible},
    numbers=left,
    xrightmargin=0zw,
    xleftmargin=3zw,
    numberstyle={\scriptsize},
    stepnumber=1,
    numbersep=1zw,
    lineskip=-0.5ex
}
\newcommand{\prog}[3]{
    \lstinputlisting[label=code:#1, caption=#2]{materials/#3}
}
\renewcommand{\lstlistingname}{コード}

%---------------------------------------------------------------------
\makeatletter
\def\Hline{
    \noalign{\ifnum0=`}\fi\hrule \@height 2pt \futurelet
    \reserved@a\@xhline
}
\makeatother

%表の空欄
\newcommand{\blank}{\textbf{---}}

%数式(番号付き)
\newcommand{\eq}[1]{
    \begin{eqnarray}
        #1
    \end{eqnarray}
}
%数式(番号無し)
\newcommand{\EQ}[1]{
    \begin{eqnarray*}
        #1
    \end{eqnarray*}
}

%一階微分
\newcommand{\diff}[2]{\frac{d #1}{d #2}}
%二階微分
\newcommand{\DIFF}[2]{\frac{d^2 #1}{d {#2}^2}}

%図(1つ)
\newcommand{\fig}[4][0.9]{
    \begin{figure}[H]
        \centering
        \includegraphics[width=#1\linewidth]{materials/#2}
        \caption{#3}
        \label{fig:#4}
    \end{figure}
}
%図(2つ)
\newcommand{\subfig}[6]{
    \begin{figure}[H]
        \centering
        \begin{minipage}{0.45\linewidth}
            \centering
            \includegraphics[width=\linewidth]{materials/#1}
            \caption{#2}
            \label{fig:#3}
        \end{minipage}
        \begin{minipage}{0.45\linewidth}
            \centering
            \includegraphics[width=\linewidth]{materials/#4}
            \caption{#5}
            \label{fig:#6}
        \end{minipage}
    \end{figure}
}
%図(3つ)
\newcommand{\thirdfig}[9]{
    \begin{figure}[H]
        \begin{minipage}{0.3\linewidth}
            \centering
            \includegraphics[width=\linewidth]{materials/#1}
            \caption{#2}
            \label{fig:#3}
        \end{minipage}
        \begin{minipage}{0.3\linewidth}
            \centering
            \includegraphics[width=\linewidth]{materials/#4}
            \caption{#5}
            \label{fig:#6}
        \end{minipage}
        \begin{minipage}{0.3\linewidth}
            \centering
            \includegraphics[width=\linewidth]{materials/#7}
            \caption{#8}
            \label{fig:#9}
        \end{minipage}
    \end{figure}
}

\pagestyle{empty}

\begin{document} 

\twocolumn[
    \centering
    {\LARGE \textbf{2-コーダルリング$CR(N, 4, *)$の独立全域木の構築}}\\
    \vspace{1zw}
    {\large E1832 藤村勇仁 \hspace{3zw} 指導教員 濱田幸弘}\\
    \vspace{2zw}
]

\section{はじめに}
    グラフの2頂点間に内素な道がn本あれば、n-1本の道が故障しても通信が行うことができる。また、データを分割して冗長性を持たせて送信することで、いくつかのデータが破損や送信失敗した場合でも受信者側でデータの復元を行うことができる。\par
    本研究では、グラフ理論を用いてデータ通信に関する問題を解決するために2-コーダルリングというグラフ上での独立全域木の構築について述べる。

\section{グラフ}
    \subsection{グラフ}
        グラフ$G$は頂点の空でない有限集合$V(G)$(頂点集合)と、$V(G)$の二要素部分集合である辺の集合$E(G)$(辺集合)のことであり、このようなグラフを$G=(V,E)$と書く。また、頂点の数を位数、辺の数をサイズといい、位数$p$、サイズ$q$のグラフを$(p,q)$グラフという。

    \subsection{道}
        $(v_{i-1}, v_i) \in E$を満たす頂点の列$P = \langle v_0, v_1, \cdots, v_n \rangle$を道といい、$n$を$P$の長さという。$v_0$を$P$の始点, $v_n$を$P$の終点といい、$v_0 = v_n$のとき$P$は閉路という。

    \subsection{独立全域木}
        どの2頂点間にも道が存在し、閉路を含まないグラフを木という。
        グラフのすべての頂点を含み木である部分グラフを全域木という。また、グラフの同じ頂点を根とする2つの全域木を考えたときに根とすべての頂点間で道が始点と終点以外で同じ頂点を通らないとき、それらは独立全域木である。

    \subsection{2-コーダルリング}
        2-コーダルリングは、次のように定義され、$CR(N, d_1, d_2)$と書かれる。
        \begin{equation*}
            \begin{split}
                G &= (V, E) \\
                V &= \{0, 1, \cdots, N-1\} \\
                E &= \{(u, v) \mid [v-u]_N = 1 \text{or} [v-u]_N = d_i, \\
                  &\qquad 1 \leqq i \leqq 2, 2 \leqq d_1 < d_2 \leqq N/2\} \\
                  &\text{ここで、}[v-u]_N \text{は} (v-u) \bmod N \text{を示す}
            \end{split}
        \end{equation*}
        図\ref{fig:chordal}に$N = 14, d_1 = 4, d_2 = 7$の2-コーダルリングを、図\ref{fig:}と図\ref{fig:}にその2-コーダルリング上の同じ頂点を根とする二つの独立全域木を示す。
      
        \fig[0.58]{chordal.pdf}{2-コーダルリングの例}{chordal}

\section{本研究について}
    \subsection{目的}
        2-コーダルリング$CR(N, 4, *)$において、6つの全域木を構築するアルゴリズムを考案し、それらが独立であることをプログラムを使い検証することが本研究の目的である。$*$は4より大きく、$N/2$以下の任意の整数を表す。

    \subsection{先行研究}
        先行研究により、以下が知られている。
        \begin{itemize}
            \item グラフがk-連結であることと、グラフの任意の2頂点間に少なくともk本の内素な道が存在することが同値である。
        \end{itemize}

    \subsection{2-コーダルリングにおける独立全域木の構築}
    

\bibliographystyle{jplain}
\bibliography{reference.bib}

\nocite{2006離散数学入門}
\nocite{chartrand1993applied}

\end{document}
