\documentclass[twocolumn, 11pt]{jsarticle}

% \usepackage[utf8]{inputenc}
\usepackage[dvipdfmx]{graphicx}
\usepackage[top=25truemm, bottom=25truemm, left=20truemm, right=20truemm]{geometry}
\usepackage{okumacro}
\usepackage{diagbox}
\usepackage{amsmath, amssymb}
\usepackage{enumerate}
\usepackage{siunitx}
\usepackage{url}
\usepackage{pdfpages}
\usepackage{here}
\usepackage{slashbox}
\usepackage{diagbox}


\setlength\textfloatsep{2truemm}

%---------------------------------------------------------------------

% \fontsize{11ptj}{16ptj}\selectfont
\setlength{\baselineskip}{16pt}
\setlength{\columnsep}{5mm}

%---------------------------------------------------------------------

%表
\usepackage{tabularx}
\newcolumntype{Y}{&gt;{\centering\arraybackslash}X}

%---------------------------------------------------------------------

\usepackage{listings,jvlisting}
\lstset{
    basicstyle={\ttfamily},
    identifierstyle={\small},
    commentstyle={\smallitshape},
    keywordstyle={\small\bfseries},
    ndkeywordstyle={\small},
    stringstyle={\small\ttfamily},
    frame={tb},
    tabsize=4,
    breaklines=true,
    columns=[l]{fullflexible},
    numbers=left,
    xrightmargin=0zw,
    xleftmargin=3zw,
    numberstyle={\scriptsize},
    stepnumber=1,
    numbersep=1zw,
    lineskip=-0.5ex
}
\newcommand{\prog}[3]{
    \lstinputlisting[label=code:#1, caption=#2]{materials/#3}
}
\renewcommand{\lstlistingname}{コード}

%---------------------------------------------------------------------
\makeatletter
\def\Hline{
    \noalign{\ifnum0=`}\fi\hrule \@height 2pt \futurelet
    \reserved@a\@xhline
}
\makeatother

%表の空欄
\newcommand{\blank}{\textbf{---}}

%数式(番号付き)
\newcommand{\eq}[1]{
    \begin{eqnarray}
        #1
    \end{eqnarray}
}
%数式(番号無し)
\newcommand{\EQ}[1]{
    \begin{eqnarray*}
        #1
    \end{eqnarray*}
}

%一階微分
\newcommand{\diff}[2]{\frac{d #1}{d #2}}
%二階微分
\newcommand{\DIFF}[2]{\frac{d^2 #1}{d {#2}^2}}

%図(1つ)
\newcommand{\fig}[4][0.9]{
    \begin{figure}[H]
        \centering
        \includegraphics[width=#1\linewidth]{materials/#2}
        \caption{#3}
        \label{fig:#4}
    \end{figure}
}
%図(2つ)
\newcommand{\subfig}[6]{
    \begin{figure}[H]
        \centering
        \begin{minipage}{0.45\linewidth}
            \centering
            \includegraphics[width=\linewidth]{materials/#1}
            \caption{#2}
            \label{fig:#3}
        \end{minipage}
        \begin{minipage}{0.45\linewidth}
            \centering
            \includegraphics[width=\linewidth]{materials/#4}
            \caption{#5}
            \label{fig:#6}
        \end{minipage}
    \end{figure}
}
%図(3つ)
\newcommand{\thirdfig}[9]{
    \begin{figure}[H]
        \begin{minipage}{0.3\linewidth}
            \centering
            \includegraphics[width=\linewidth]{materials/#1}
            \caption{#2}
            \label{fig:#3}
        \end{minipage}
        \begin{minipage}{0.3\linewidth}
            \centering
            \includegraphics[width=\linewidth]{materials/#4}
            \caption{#5}
            \label{fig:#6}
        \end{minipage}
        \begin{minipage}{0.3\linewidth}
            \centering
            \includegraphics[width=\linewidth]{materials/#7}
            \caption{#8}
            \label{fig:#9}
        \end{minipage}
    \end{figure}
}

\pagestyle{empty}

\begin{document} 

\twocolumn[
    \centering
    {\LARGE \textbf{2-コーダルリング$CR(N, 4, *)$の独立全域木の構築}}\\
    \vspace{1zw}
    {\large E1832 藤村勇仁 \hspace{3zw} 指導教員 濱田幸弘}\\
    \vspace{2zw}
]

\section{はじめに}
    グラフ理論とは、ネットワークや交通などの様々な問題をグラフを用いて表現することで、グラフの持つ性質からもとの問題の解析を行う分野である。本研究ではグラフ理論をデータ通信に関する問題を解決するために用いる。
    グラフの2頂点間に内素な道が$n$本あれば、$n-1$本の道が故障しても通信が行うことができる。また、データを分割して冗長性を持たせて送信することで、いくつかのデータが破損や送信失敗した場合でも受信者側でデータの復元を行うことができる。
    本研究では、2頂点間の内素な道を確保するために2-コーダルリングというグラフ上での独立全域木の構築に取り組む。

\section{グラフ}
    \subsection{グラフ}
        グラフ$G$は頂点の空でない有限集合$V(G)$と、$V(G)$の2要素部分集合である辺の集合$E(G)$の組であり、形式的に$G=(V,E)$と書く。また、頂点の数を位数、辺の数をサイズといい、位数$p$、サイズ$q$のグラフを$(p,q)$グラフという。
        $e = uv$がグラフ$G$の辺であるとき、$u$と$v$は隣接していると言う\cite{2006離散数学入門}。
    
    \subsection{道}
        $v_{i-1}v_i \in E$を満たす頂点の列$P = \langle v_0, v_1, \cdots, v_n \rangle$を道といい、$n$を$P$の長さという。$v_0$を$P$の始点, $v_n$を$P$の終点といい、$v_0 = v_n$のとき$P$は閉路という\cite{2006離散数学入門}。

    \subsection{内素}
       始点と終点が等しい 複数の道が内素であるとは、どの2つの道も始点と終点以外で同じ頂点と辺を通っていないことをいう\cite{chartrand1993applied}。
    
    \subsection{木}
        閉路を含まない連結なグラフを木という\cite{2006離散数学入門}。
        
    \subsection{全域部分グラフ}
        グラフ$G=(V,E)$に対し、$V' = V, E' \subseteqq E$であるようなグラフ$G'=(V', E')$を$G$の全域部分グラフという。

    \subsection{全域木}
        連結グラフ$G$の全域部分グラフ$G'$が木であるなら、$G'$のことを全域木と呼ぶ\cite{2006離散数学入門}。

    \subsection{独立全域木}
        グラフ$G$において同一の頂点を根とする複数の全域木について、根を始点とし根を除く任意の頂点を終点とする道が自分以外の全ての全域木について内素であるとき、これらの全域木は独立であるという\cite{chartrand1993applied}。

    \subsection{2-コーダルリング}
        $G = (V, E)$は下のように定義されるとき、2-コーダルリングと呼ばれ、$CR(N, d_1, d_2)$と書かれる\cite{YukihiroHAMADA2016}。
        \begin{equation*}
            \begin{split}
                V &= \{0, 1, \cdots, N-1\} \\
                E &= \{(u, v) \mid [v-u]_N = 1 \text{または} [v-u]_N = d_1, \\
                  &\text{または} [v-u]_N = d_2, 2 \leqq d_1 < d_2 \leqq N/2\} \\
                  &\text{ここで、}[v-u]_N \text{は} (v-u) \bmod N \text{を表す。}
            \end{split}
        \end{equation*}
        2-コーダルリングは、$N\geq7, d_2\neq\frac{N}{2}$のとき$CR(N,d_1,d_2)$は6-正則である。
        図\ref{fig:chordal}に$N = 14, d_1 = 4, d_2 = 7$の2-コーダルリングを、図\ref{fig:IST}にその2-コーダルリング上の頂点0を根とする2つの独立全域木を示す。
        \newpage
      
        \fig[0.5]{list.pdf}{$CR(12,4,5)$}{chordal}
        \begin{figure}[H]
            \centering
            \begin{minipage}{0.45\linewidth}
                \centering
                \includegraphics[width=\linewidth]{figures/IST1.pdf}
            \end{minipage}
            \begin{minipage}{0.45\linewidth}
                \centering
                \includegraphics[width=\linewidth]{figures/graphs-ad.pdf}
            \end{minipage}
            \caption{$CR(14,4,7)$の2つの独立全域木}
            \label{fig:IST}
        \end{figure}

\section{卒業研究について}
    \subsection{先行研究}
        先行研究により、以下が知られている。
        \begin{itemize}
            \item グラフが$k$-連結であることと、グラフの任意の2頂点間に少なくとも$k$本の内素な道が存在することが同値である\cite{YukihiroHAMADA2016}。それにより、2-コーダルリングについても同様に$k$-連結であれば$k$本の内素な道が存在するが、一般的な構築方法は知られていない。ここでは、2-コーダルリングについてのみ記述する。
            \item 以下に示す6連結のコーダルリング$CR(N,d_1,d_2)$において6つの全域木を構築するアルゴリズムが考案され、それらが独立であることがプログラムを使い検証されている\cite{Yokooji}。
                \begin{itemize}
                    \item $N\geqq3d_2-2, d_1=2, 2<d_2<N/2$
                    \item $N\geqq17, 3\leqq d_1\leqq\frac{N-1}{4}, d_2=2d_1-1$
                    \item $N\geqq10, 2\leqq d_1\leqq\frac{N}{5}, d_2=2d_1$
                    \item $N\geqq9, 2\leqq d_1<\frac{N}{4}, d_2=2d_1$
                    \item $N\geqq5, 2\leqq d_1\leqq\frac{N-1}{3}, d_2=d_1+1$
                \end{itemize}
        \end{itemize}
        
    \subsection{研究の目的}
        2-コーダルリング$CR(N, 4, *)$において、6つの全域木を構築するアルゴリズムを考案し、それらが独立であることをプログラムを使い検証することが本研究の目的である。ここで、$*$は4より大きく、$N/2$より小さい任意の整数を表す。先行研究にでは、$CR(N, 4, *)$の独立全域木の構築方法が知られていない。

    \subsection{進捗状況}
        4年次の課題研究、及び5年次の半年間でグラフ理論に関する資料や先行研究の論文を読んで研究対象に関する知識を身に着けた。

    \subsection{今後の方針}
        2-コーダルリング$CR(N,4,*)$上で$*$にいくつかの値をあてはめ、6つの全域木を書くことより規則性を模索したうえで、それらを構築するアルゴリズムを考案する。その後、それらが独立であるかをプログラムを作成し検証する。
    
    
\bibliographystyle{jplain}
\bibliography{reference.bib}

\end{document}
